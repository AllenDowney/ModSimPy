
% Default to the notebook output style

    


% Inherit from the specified cell style.




    
\documentclass[11pt]{article}

    
    
    \usepackage[T1]{fontenc}
    % Nicer default font (+ math font) than Computer Modern for most use cases
    \usepackage{mathpazo}

    % Basic figure setup, for now with no caption control since it's done
    % automatically by Pandoc (which extracts ![](path) syntax from Markdown).
    \usepackage{graphicx}
    % We will generate all images so they have a width \maxwidth. This means
    % that they will get their normal width if they fit onto the page, but
    % are scaled down if they would overflow the margins.
    \makeatletter
    \def\maxwidth{\ifdim\Gin@nat@width>\linewidth\linewidth
    \else\Gin@nat@width\fi}
    \makeatother
    \let\Oldincludegraphics\includegraphics
    % Set max figure width to be 80% of text width, for now hardcoded.
    \renewcommand{\includegraphics}[1]{\Oldincludegraphics[width=.8\maxwidth]{#1}}
    % Ensure that by default, figures have no caption (until we provide a
    % proper Figure object with a Caption API and a way to capture that
    % in the conversion process - todo).
    \usepackage{caption}
    \DeclareCaptionLabelFormat{nolabel}{}
    \captionsetup{labelformat=nolabel}

    \usepackage{adjustbox} % Used to constrain images to a maximum size 
    \usepackage{xcolor} % Allow colors to be defined
    \usepackage{enumerate} % Needed for markdown enumerations to work
    \usepackage{geometry} % Used to adjust the document margins
    \usepackage{amsmath} % Equations
    \usepackage{amssymb} % Equations
    \usepackage{textcomp} % defines textquotesingle
    % Hack from http://tex.stackexchange.com/a/47451/13684:
    \AtBeginDocument{%
        \def\PYZsq{\textquotesingle}% Upright quotes in Pygmentized code
    }
    \usepackage{upquote} % Upright quotes for verbatim code
    \usepackage{eurosym} % defines \euro
    \usepackage[mathletters]{ucs} % Extended unicode (utf-8) support
    \usepackage[utf8x]{inputenc} % Allow utf-8 characters in the tex document
    \usepackage{fancyvrb} % verbatim replacement that allows latex
    \usepackage{grffile} % extends the file name processing of package graphics 
                         % to support a larger range 
    % The hyperref package gives us a pdf with properly built
    % internal navigation ('pdf bookmarks' for the table of contents,
    % internal cross-reference links, web links for URLs, etc.)
    \usepackage{hyperref}
    \usepackage{longtable} % longtable support required by pandoc >1.10
    \usepackage{booktabs}  % table support for pandoc > 1.12.2
    \usepackage[inline]{enumitem} % IRkernel/repr support (it uses the enumerate* environment)
    \usepackage[normalem]{ulem} % ulem is needed to support strikethroughs (\sout)
                                % normalem makes italics be italics, not underlines
    

    
    
    % Colors for the hyperref package
    \definecolor{urlcolor}{rgb}{0,.145,.698}
    \definecolor{linkcolor}{rgb}{.71,0.21,0.01}
    \definecolor{citecolor}{rgb}{.12,.54,.11}

    % ANSI colors
    \definecolor{ansi-black}{HTML}{3E424D}
    \definecolor{ansi-black-intense}{HTML}{282C36}
    \definecolor{ansi-red}{HTML}{E75C58}
    \definecolor{ansi-red-intense}{HTML}{B22B31}
    \definecolor{ansi-green}{HTML}{00A250}
    \definecolor{ansi-green-intense}{HTML}{007427}
    \definecolor{ansi-yellow}{HTML}{DDB62B}
    \definecolor{ansi-yellow-intense}{HTML}{B27D12}
    \definecolor{ansi-blue}{HTML}{208FFB}
    \definecolor{ansi-blue-intense}{HTML}{0065CA}
    \definecolor{ansi-magenta}{HTML}{D160C4}
    \definecolor{ansi-magenta-intense}{HTML}{A03196}
    \definecolor{ansi-cyan}{HTML}{60C6C8}
    \definecolor{ansi-cyan-intense}{HTML}{258F8F}
    \definecolor{ansi-white}{HTML}{C5C1B4}
    \definecolor{ansi-white-intense}{HTML}{A1A6B2}

    % commands and environments needed by pandoc snippets
    % extracted from the output of `pandoc -s`
    \providecommand{\tightlist}{%
      \setlength{\itemsep}{0pt}\setlength{\parskip}{0pt}}
    \DefineVerbatimEnvironment{Highlighting}{Verbatim}{commandchars=\\\{\}}
    % Add ',fontsize=\small' for more characters per line
    \newenvironment{Shaded}{}{}
    \newcommand{\KeywordTok}[1]{\textcolor[rgb]{0.00,0.44,0.13}{\textbf{{#1}}}}
    \newcommand{\DataTypeTok}[1]{\textcolor[rgb]{0.56,0.13,0.00}{{#1}}}
    \newcommand{\DecValTok}[1]{\textcolor[rgb]{0.25,0.63,0.44}{{#1}}}
    \newcommand{\BaseNTok}[1]{\textcolor[rgb]{0.25,0.63,0.44}{{#1}}}
    \newcommand{\FloatTok}[1]{\textcolor[rgb]{0.25,0.63,0.44}{{#1}}}
    \newcommand{\CharTok}[1]{\textcolor[rgb]{0.25,0.44,0.63}{{#1}}}
    \newcommand{\StringTok}[1]{\textcolor[rgb]{0.25,0.44,0.63}{{#1}}}
    \newcommand{\CommentTok}[1]{\textcolor[rgb]{0.38,0.63,0.69}{\textit{{#1}}}}
    \newcommand{\OtherTok}[1]{\textcolor[rgb]{0.00,0.44,0.13}{{#1}}}
    \newcommand{\AlertTok}[1]{\textcolor[rgb]{1.00,0.00,0.00}{\textbf{{#1}}}}
    \newcommand{\FunctionTok}[1]{\textcolor[rgb]{0.02,0.16,0.49}{{#1}}}
    \newcommand{\RegionMarkerTok}[1]{{#1}}
    \newcommand{\ErrorTok}[1]{\textcolor[rgb]{1.00,0.00,0.00}{\textbf{{#1}}}}
    \newcommand{\NormalTok}[1]{{#1}}
    
    % Additional commands for more recent versions of Pandoc
    \newcommand{\ConstantTok}[1]{\textcolor[rgb]{0.53,0.00,0.00}{{#1}}}
    \newcommand{\SpecialCharTok}[1]{\textcolor[rgb]{0.25,0.44,0.63}{{#1}}}
    \newcommand{\VerbatimStringTok}[1]{\textcolor[rgb]{0.25,0.44,0.63}{{#1}}}
    \newcommand{\SpecialStringTok}[1]{\textcolor[rgb]{0.73,0.40,0.53}{{#1}}}
    \newcommand{\ImportTok}[1]{{#1}}
    \newcommand{\DocumentationTok}[1]{\textcolor[rgb]{0.73,0.13,0.13}{\textit{{#1}}}}
    \newcommand{\AnnotationTok}[1]{\textcolor[rgb]{0.38,0.63,0.69}{\textbf{\textit{{#1}}}}}
    \newcommand{\CommentVarTok}[1]{\textcolor[rgb]{0.38,0.63,0.69}{\textbf{\textit{{#1}}}}}
    \newcommand{\VariableTok}[1]{\textcolor[rgb]{0.10,0.09,0.49}{{#1}}}
    \newcommand{\ControlFlowTok}[1]{\textcolor[rgb]{0.00,0.44,0.13}{\textbf{{#1}}}}
    \newcommand{\OperatorTok}[1]{\textcolor[rgb]{0.40,0.40,0.40}{{#1}}}
    \newcommand{\BuiltInTok}[1]{{#1}}
    \newcommand{\ExtensionTok}[1]{{#1}}
    \newcommand{\PreprocessorTok}[1]{\textcolor[rgb]{0.74,0.48,0.00}{{#1}}}
    \newcommand{\AttributeTok}[1]{\textcolor[rgb]{0.49,0.56,0.16}{{#1}}}
    \newcommand{\InformationTok}[1]{\textcolor[rgb]{0.38,0.63,0.69}{\textbf{\textit{{#1}}}}}
    \newcommand{\WarningTok}[1]{\textcolor[rgb]{0.38,0.63,0.69}{\textbf{\textit{{#1}}}}}
    
    
    % Define a nice break command that doesn't care if a line doesn't already
    % exist.
    \def\br{\hspace*{\fill} \\* }
    % Math Jax compatability definitions
    \def\gt{>}
    \def\lt{<}
    % Document parameters
    \title{project1\_final\_V1.0}
    
    
    

    % Pygments definitions
    
\makeatletter
\def\PY@reset{\let\PY@it=\relax \let\PY@bf=\relax%
    \let\PY@ul=\relax \let\PY@tc=\relax%
    \let\PY@bc=\relax \let\PY@ff=\relax}
\def\PY@tok#1{\csname PY@tok@#1\endcsname}
\def\PY@toks#1+{\ifx\relax#1\empty\else%
    \PY@tok{#1}\expandafter\PY@toks\fi}
\def\PY@do#1{\PY@bc{\PY@tc{\PY@ul{%
    \PY@it{\PY@bf{\PY@ff{#1}}}}}}}
\def\PY#1#2{\PY@reset\PY@toks#1+\relax+\PY@do{#2}}

\expandafter\def\csname PY@tok@w\endcsname{\def\PY@tc##1{\textcolor[rgb]{0.73,0.73,0.73}{##1}}}
\expandafter\def\csname PY@tok@c\endcsname{\let\PY@it=\textit\def\PY@tc##1{\textcolor[rgb]{0.25,0.50,0.50}{##1}}}
\expandafter\def\csname PY@tok@cp\endcsname{\def\PY@tc##1{\textcolor[rgb]{0.74,0.48,0.00}{##1}}}
\expandafter\def\csname PY@tok@k\endcsname{\let\PY@bf=\textbf\def\PY@tc##1{\textcolor[rgb]{0.00,0.50,0.00}{##1}}}
\expandafter\def\csname PY@tok@kp\endcsname{\def\PY@tc##1{\textcolor[rgb]{0.00,0.50,0.00}{##1}}}
\expandafter\def\csname PY@tok@kt\endcsname{\def\PY@tc##1{\textcolor[rgb]{0.69,0.00,0.25}{##1}}}
\expandafter\def\csname PY@tok@o\endcsname{\def\PY@tc##1{\textcolor[rgb]{0.40,0.40,0.40}{##1}}}
\expandafter\def\csname PY@tok@ow\endcsname{\let\PY@bf=\textbf\def\PY@tc##1{\textcolor[rgb]{0.67,0.13,1.00}{##1}}}
\expandafter\def\csname PY@tok@nb\endcsname{\def\PY@tc##1{\textcolor[rgb]{0.00,0.50,0.00}{##1}}}
\expandafter\def\csname PY@tok@nf\endcsname{\def\PY@tc##1{\textcolor[rgb]{0.00,0.00,1.00}{##1}}}
\expandafter\def\csname PY@tok@nc\endcsname{\let\PY@bf=\textbf\def\PY@tc##1{\textcolor[rgb]{0.00,0.00,1.00}{##1}}}
\expandafter\def\csname PY@tok@nn\endcsname{\let\PY@bf=\textbf\def\PY@tc##1{\textcolor[rgb]{0.00,0.00,1.00}{##1}}}
\expandafter\def\csname PY@tok@ne\endcsname{\let\PY@bf=\textbf\def\PY@tc##1{\textcolor[rgb]{0.82,0.25,0.23}{##1}}}
\expandafter\def\csname PY@tok@nv\endcsname{\def\PY@tc##1{\textcolor[rgb]{0.10,0.09,0.49}{##1}}}
\expandafter\def\csname PY@tok@no\endcsname{\def\PY@tc##1{\textcolor[rgb]{0.53,0.00,0.00}{##1}}}
\expandafter\def\csname PY@tok@nl\endcsname{\def\PY@tc##1{\textcolor[rgb]{0.63,0.63,0.00}{##1}}}
\expandafter\def\csname PY@tok@ni\endcsname{\let\PY@bf=\textbf\def\PY@tc##1{\textcolor[rgb]{0.60,0.60,0.60}{##1}}}
\expandafter\def\csname PY@tok@na\endcsname{\def\PY@tc##1{\textcolor[rgb]{0.49,0.56,0.16}{##1}}}
\expandafter\def\csname PY@tok@nt\endcsname{\let\PY@bf=\textbf\def\PY@tc##1{\textcolor[rgb]{0.00,0.50,0.00}{##1}}}
\expandafter\def\csname PY@tok@nd\endcsname{\def\PY@tc##1{\textcolor[rgb]{0.67,0.13,1.00}{##1}}}
\expandafter\def\csname PY@tok@s\endcsname{\def\PY@tc##1{\textcolor[rgb]{0.73,0.13,0.13}{##1}}}
\expandafter\def\csname PY@tok@sd\endcsname{\let\PY@it=\textit\def\PY@tc##1{\textcolor[rgb]{0.73,0.13,0.13}{##1}}}
\expandafter\def\csname PY@tok@si\endcsname{\let\PY@bf=\textbf\def\PY@tc##1{\textcolor[rgb]{0.73,0.40,0.53}{##1}}}
\expandafter\def\csname PY@tok@se\endcsname{\let\PY@bf=\textbf\def\PY@tc##1{\textcolor[rgb]{0.73,0.40,0.13}{##1}}}
\expandafter\def\csname PY@tok@sr\endcsname{\def\PY@tc##1{\textcolor[rgb]{0.73,0.40,0.53}{##1}}}
\expandafter\def\csname PY@tok@ss\endcsname{\def\PY@tc##1{\textcolor[rgb]{0.10,0.09,0.49}{##1}}}
\expandafter\def\csname PY@tok@sx\endcsname{\def\PY@tc##1{\textcolor[rgb]{0.00,0.50,0.00}{##1}}}
\expandafter\def\csname PY@tok@m\endcsname{\def\PY@tc##1{\textcolor[rgb]{0.40,0.40,0.40}{##1}}}
\expandafter\def\csname PY@tok@gh\endcsname{\let\PY@bf=\textbf\def\PY@tc##1{\textcolor[rgb]{0.00,0.00,0.50}{##1}}}
\expandafter\def\csname PY@tok@gu\endcsname{\let\PY@bf=\textbf\def\PY@tc##1{\textcolor[rgb]{0.50,0.00,0.50}{##1}}}
\expandafter\def\csname PY@tok@gd\endcsname{\def\PY@tc##1{\textcolor[rgb]{0.63,0.00,0.00}{##1}}}
\expandafter\def\csname PY@tok@gi\endcsname{\def\PY@tc##1{\textcolor[rgb]{0.00,0.63,0.00}{##1}}}
\expandafter\def\csname PY@tok@gr\endcsname{\def\PY@tc##1{\textcolor[rgb]{1.00,0.00,0.00}{##1}}}
\expandafter\def\csname PY@tok@ge\endcsname{\let\PY@it=\textit}
\expandafter\def\csname PY@tok@gs\endcsname{\let\PY@bf=\textbf}
\expandafter\def\csname PY@tok@gp\endcsname{\let\PY@bf=\textbf\def\PY@tc##1{\textcolor[rgb]{0.00,0.00,0.50}{##1}}}
\expandafter\def\csname PY@tok@go\endcsname{\def\PY@tc##1{\textcolor[rgb]{0.53,0.53,0.53}{##1}}}
\expandafter\def\csname PY@tok@gt\endcsname{\def\PY@tc##1{\textcolor[rgb]{0.00,0.27,0.87}{##1}}}
\expandafter\def\csname PY@tok@err\endcsname{\def\PY@bc##1{\setlength{\fboxsep}{0pt}\fcolorbox[rgb]{1.00,0.00,0.00}{1,1,1}{\strut ##1}}}
\expandafter\def\csname PY@tok@kc\endcsname{\let\PY@bf=\textbf\def\PY@tc##1{\textcolor[rgb]{0.00,0.50,0.00}{##1}}}
\expandafter\def\csname PY@tok@kd\endcsname{\let\PY@bf=\textbf\def\PY@tc##1{\textcolor[rgb]{0.00,0.50,0.00}{##1}}}
\expandafter\def\csname PY@tok@kn\endcsname{\let\PY@bf=\textbf\def\PY@tc##1{\textcolor[rgb]{0.00,0.50,0.00}{##1}}}
\expandafter\def\csname PY@tok@kr\endcsname{\let\PY@bf=\textbf\def\PY@tc##1{\textcolor[rgb]{0.00,0.50,0.00}{##1}}}
\expandafter\def\csname PY@tok@bp\endcsname{\def\PY@tc##1{\textcolor[rgb]{0.00,0.50,0.00}{##1}}}
\expandafter\def\csname PY@tok@fm\endcsname{\def\PY@tc##1{\textcolor[rgb]{0.00,0.00,1.00}{##1}}}
\expandafter\def\csname PY@tok@vc\endcsname{\def\PY@tc##1{\textcolor[rgb]{0.10,0.09,0.49}{##1}}}
\expandafter\def\csname PY@tok@vg\endcsname{\def\PY@tc##1{\textcolor[rgb]{0.10,0.09,0.49}{##1}}}
\expandafter\def\csname PY@tok@vi\endcsname{\def\PY@tc##1{\textcolor[rgb]{0.10,0.09,0.49}{##1}}}
\expandafter\def\csname PY@tok@vm\endcsname{\def\PY@tc##1{\textcolor[rgb]{0.10,0.09,0.49}{##1}}}
\expandafter\def\csname PY@tok@sa\endcsname{\def\PY@tc##1{\textcolor[rgb]{0.73,0.13,0.13}{##1}}}
\expandafter\def\csname PY@tok@sb\endcsname{\def\PY@tc##1{\textcolor[rgb]{0.73,0.13,0.13}{##1}}}
\expandafter\def\csname PY@tok@sc\endcsname{\def\PY@tc##1{\textcolor[rgb]{0.73,0.13,0.13}{##1}}}
\expandafter\def\csname PY@tok@dl\endcsname{\def\PY@tc##1{\textcolor[rgb]{0.73,0.13,0.13}{##1}}}
\expandafter\def\csname PY@tok@s2\endcsname{\def\PY@tc##1{\textcolor[rgb]{0.73,0.13,0.13}{##1}}}
\expandafter\def\csname PY@tok@sh\endcsname{\def\PY@tc##1{\textcolor[rgb]{0.73,0.13,0.13}{##1}}}
\expandafter\def\csname PY@tok@s1\endcsname{\def\PY@tc##1{\textcolor[rgb]{0.73,0.13,0.13}{##1}}}
\expandafter\def\csname PY@tok@mb\endcsname{\def\PY@tc##1{\textcolor[rgb]{0.40,0.40,0.40}{##1}}}
\expandafter\def\csname PY@tok@mf\endcsname{\def\PY@tc##1{\textcolor[rgb]{0.40,0.40,0.40}{##1}}}
\expandafter\def\csname PY@tok@mh\endcsname{\def\PY@tc##1{\textcolor[rgb]{0.40,0.40,0.40}{##1}}}
\expandafter\def\csname PY@tok@mi\endcsname{\def\PY@tc##1{\textcolor[rgb]{0.40,0.40,0.40}{##1}}}
\expandafter\def\csname PY@tok@il\endcsname{\def\PY@tc##1{\textcolor[rgb]{0.40,0.40,0.40}{##1}}}
\expandafter\def\csname PY@tok@mo\endcsname{\def\PY@tc##1{\textcolor[rgb]{0.40,0.40,0.40}{##1}}}
\expandafter\def\csname PY@tok@ch\endcsname{\let\PY@it=\textit\def\PY@tc##1{\textcolor[rgb]{0.25,0.50,0.50}{##1}}}
\expandafter\def\csname PY@tok@cm\endcsname{\let\PY@it=\textit\def\PY@tc##1{\textcolor[rgb]{0.25,0.50,0.50}{##1}}}
\expandafter\def\csname PY@tok@cpf\endcsname{\let\PY@it=\textit\def\PY@tc##1{\textcolor[rgb]{0.25,0.50,0.50}{##1}}}
\expandafter\def\csname PY@tok@c1\endcsname{\let\PY@it=\textit\def\PY@tc##1{\textcolor[rgb]{0.25,0.50,0.50}{##1}}}
\expandafter\def\csname PY@tok@cs\endcsname{\let\PY@it=\textit\def\PY@tc##1{\textcolor[rgb]{0.25,0.50,0.50}{##1}}}

\def\PYZbs{\char`\\}
\def\PYZus{\char`\_}
\def\PYZob{\char`\{}
\def\PYZcb{\char`\}}
\def\PYZca{\char`\^}
\def\PYZam{\char`\&}
\def\PYZlt{\char`\<}
\def\PYZgt{\char`\>}
\def\PYZsh{\char`\#}
\def\PYZpc{\char`\%}
\def\PYZdl{\char`\$}
\def\PYZhy{\char`\-}
\def\PYZsq{\char`\'}
\def\PYZdq{\char`\"}
\def\PYZti{\char`\~}
% for compatibility with earlier versions
\def\PYZat{@}
\def\PYZlb{[}
\def\PYZrb{]}
\makeatother


    % Exact colors from NB
    \definecolor{incolor}{rgb}{0.0, 0.0, 0.5}
    \definecolor{outcolor}{rgb}{0.545, 0.0, 0.0}



    
    % Prevent overflowing lines due to hard-to-break entities
    \sloppy 
    % Setup hyperref package
    \hypersetup{
      breaklinks=true,  % so long urls are correctly broken across lines
      colorlinks=true,
      urlcolor=urlcolor,
      linkcolor=linkcolor,
      citecolor=citecolor,
      }
    % Slightly bigger margins than the latex defaults
    
    \geometry{verbose,tmargin=1in,bmargin=1in,lmargin=1in,rmargin=1in}
    
    

    \begin{document}
    
    
    \maketitle
    
    

    
    \begin{Verbatim}[commandchars=\\\{\}]
{\color{incolor}In [{\color{incolor}18}]:} \PY{o}{\PYZpc{}\PYZpc{}}\PY{k}{html}
         \PYZlt{}style\PYZgt{}
         body \PYZob{}
             font\PYZhy{}family: \PYZdq{}Helvetica Neue\PYZdq{};
         \PYZcb{}
         \PYZlt{}/style\PYZgt{}
\end{Verbatim}


    
    \begin{verbatim}
<IPython.core.display.HTML object>
    \end{verbatim}

    
    \hypertarget{project-1-modeling-populations}{%
\section{Project 1: Modeling
Populations}\label{project-1-modeling-populations}}

Alex Wenstrup and Jasmine Kamdar

September 27, 2018

    \hypertarget{questions}{%
\subsection{Questions:}\label{questions}}

\begin{itemize}
\tightlist
\item
  Why was there a population rise in Germany between 1950-1970?

  \begin{itemize}
  \tightlist
  \item
    What was the effect on the population of Portugal (one of the 7
    locations where migrant workers were coming to Germany from) during
    this time?
  \end{itemize}
\end{itemize}

    \hypertarget{methodology}{%
\subsection{Methodology:}\label{methodology}}

\begin{itemize}
\tightlist
\item
  Overlay a generic logistic growth model on top of the German
  population in the specified date range
\item
  Calculate and plot residuals to identify where the greatest deviations
  from logistic growth occurred
\item
  Research significant events in Germany within the specified date range
\item
  Add data and model for other affected population as a way to validate
  our results
\item
  Adjust models for both populations to account for found events
\item
  Plot residuals for new models, compare to residuals for old model, and
  evaluate effectiveness of new model
\end{itemize}

    \hypertarget{note-this-report-includes-all-processes-done-for-the-portuguese-population-inline-with-those-done-for-the-german-population.-that-is-not-indicative-of-our-actual-order-of-operations-rather-it-is-done-for-continuity-in-this-report.}{%
\subsubsection{Note: This report includes all processes done for the
Portuguese population inline with those done for the German population.
That is not indicative of our actual order of operations, rather it is
done for continuity in this
report.}\label{note-this-report-includes-all-processes-done-for-the-portuguese-population-inline-with-those-done-for-the-german-population.-that-is-not-indicative-of-our-actual-order-of-operations-rather-it-is-done-for-continuity-in-this-report.}}

    \hypertarget{step-1-find-and-import-data}{%
\subsection{Step 1: Find and Import
data}\label{step-1-find-and-import-data}}

We obtained our data from gapminder.com, an established resource and
tool for tracking world demographic data. After downloading their world
population data as an excel spreadsheet, we removed unnecesary rows and
columns so that our CSV file contained only data relevant to our model.
Source: https://www.gapminder.org/data/

    \begin{Verbatim}[commandchars=\\\{\}]
{\color{incolor}In [{\color{incolor}19}]:} \PY{k+kn}{import} \PY{n+nn}{pandas}
         \PY{k+kn}{from} \PY{n+nn}{modsim} \PY{k}{import} \PY{o}{*}
\end{Verbatim}


    \begin{Verbatim}[commandchars=\\\{\}]
{\color{incolor}In [{\color{incolor}20}]:} \PY{n}{rows} \PY{o}{=} \PY{l+m+mi}{64}
         
         \PY{n}{ger\PYZus{}port\PYZus{}data} \PY{o}{=} \PY{n}{pandas}\PY{o}{.}\PY{n}{read\PYZus{}csv}\PY{p}{(}\PY{l+s+s1}{\PYZsq{}}\PY{l+s+s1}{germany\PYZus{}portugal.csv}\PY{l+s+s1}{\PYZsq{}}\PY{p}{,} 
                                        \PY{n}{low\PYZus{}memory}\PY{o}{=}\PY{k+kc}{False}\PY{p}{,} 
                                        \PY{n}{usecols}\PY{o}{=}\PY{p}{[}\PY{l+m+mi}{0}\PY{p}{,} \PY{l+m+mi}{1}\PY{p}{,} \PY{l+m+mi}{2}\PY{p}{]}\PY{p}{,} 
                                        \PY{n}{nrows}\PY{o}{=}\PY{n}{rows}\PY{p}{,} 
                                        \PY{n}{index\PYZus{}col}\PY{o}{=}\PY{l+m+mi}{0}\PY{p}{)}
         
         \PY{c+c1}{\PYZsh{}Import data from saved CSV file}
         \PY{c+c1}{\PYZsh{}Note that Portuguese data was not imported originally, only later in the project}
\end{Verbatim}


    \begin{Verbatim}[commandchars=\\\{\}]
{\color{incolor}In [{\color{incolor}21}]:} \PY{n}{germany} \PY{o}{=} \PY{n}{ger\PYZus{}port\PYZus{}data}\PY{o}{.}\PY{n}{Germany} \PY{o}{/} \PY{l+m+mf}{1e6}
         \PY{n}{portugal} \PY{o}{=} \PY{n}{ger\PYZus{}port\PYZus{}data}\PY{o}{.}\PY{n}{Portugal} \PY{o}{/} \PY{l+m+mf}{1e6}
         
         \PY{c+c1}{\PYZsh{}Save data from DataFrame, in millions}
\end{Verbatim}


    \begin{Verbatim}[commandchars=\\\{\}]
{\color{incolor}In [{\color{incolor}22}]:} \PY{n}{plot}\PY{p}{(}\PY{n}{germany}\PY{p}{,} \PY{n}{label} \PY{o}{=} \PY{l+s+s1}{\PYZsq{}}\PY{l+s+s1}{German Population}\PY{l+s+s1}{\PYZsq{}}\PY{p}{)}
         
         \PY{n}{decorate}\PY{p}{(}\PY{n}{xlabel} \PY{o}{=} \PY{l+s+s1}{\PYZsq{}}\PY{l+s+s1}{Time (years)}\PY{l+s+s1}{\PYZsq{}}\PY{p}{,} \PY{n}{ylabel} \PY{o}{=} \PY{l+s+s1}{\PYZsq{}}\PY{l+s+s1}{Population (millions)}\PY{l+s+s1}{\PYZsq{}}\PY{p}{)}
         
         \PY{c+c1}{\PYZsh{}Plot of Germany\PYZsq{}s population over time}
\end{Verbatim}


    \begin{center}
    \adjustimage{max size={0.9\linewidth}{0.9\paperheight}}{output_9_0.png}
    \end{center}
    { \hspace*{\fill} \\}
    
    \begin{Verbatim}[commandchars=\\\{\}]
{\color{incolor}In [{\color{incolor}23}]:} \PY{n}{plot}\PY{p}{(}\PY{n}{portugal}\PY{p}{,} \PY{n}{label} \PY{o}{=} \PY{l+s+s1}{\PYZsq{}}\PY{l+s+s1}{Portuguese Population}\PY{l+s+s1}{\PYZsq{}}\PY{p}{)}
         
         \PY{n}{decorate}\PY{p}{(}\PY{n}{xlabel} \PY{o}{=} \PY{l+s+s1}{\PYZsq{}}\PY{l+s+s1}{Time (years)}\PY{l+s+s1}{\PYZsq{}}\PY{p}{,} \PY{n}{ylabel} \PY{o}{=} \PY{l+s+s1}{\PYZsq{}}\PY{l+s+s1}{Population (millions)}\PY{l+s+s1}{\PYZsq{}}\PY{p}{)}
         
         \PY{c+c1}{\PYZsh{}Plot of Portugal\PYZsq{}s population over time}
\end{Verbatim}


    \begin{center}
    \adjustimage{max size={0.9\linewidth}{0.9\paperheight}}{output_10_0.png}
    \end{center}
    { \hspace*{\fill} \\}
    
    \hypertarget{step-2-overlay-a-generic-growth-model}{%
\subsection{Step 2: Overlay a generic growth
model}\label{step-2-overlay-a-generic-growth-model}}

We had to pick some generic growth model from which to calculate the
largest deviations from expected population. For this purpose, we chose
a simple quadratic growth model because of its face validity, easy
workability, and general good fit.

    \begin{Verbatim}[commandchars=\\\{\}]
{\color{incolor}In [{\color{incolor}24}]:} \PY{n}{tg\PYZus{}0}\PY{o}{=}\PY{n}{get\PYZus{}first\PYZus{}label}\PY{p}{(}\PY{n}{germany}\PY{p}{)} \PY{c+c1}{\PYZsh{}Beginning of German data}
         \PY{n}{tg\PYZus{}end}\PY{o}{=}\PY{n}{get\PYZus{}last\PYZus{}label}\PY{p}{(}\PY{n}{germany}\PY{p}{)} \PY{c+c1}{\PYZsh{}End of German Data}
         \PY{n}{pg\PYZus{}0}\PY{o}{=}\PY{n}{get\PYZus{}first\PYZus{}value}\PY{p}{(}\PY{n}{germany}\PY{p}{)} \PY{c+c1}{\PYZsh{}German opulation at tg\PYZus{}0}
         
         \PY{n}{tp\PYZus{}0}\PY{o}{=}\PY{n}{get\PYZus{}first\PYZus{}label}\PY{p}{(}\PY{n}{portugal}\PY{p}{)} \PY{c+c1}{\PYZsh{}Beginning on Portuguese data}
         \PY{n}{tp\PYZus{}end}\PY{o}{=}\PY{n}{get\PYZus{}last\PYZus{}label}\PY{p}{(}\PY{n}{portugal}\PY{p}{)} \PY{c+c1}{\PYZsh{}End of Portuguese data}
         \PY{n}{pp\PYZus{}0}\PY{o}{=}\PY{n}{get\PYZus{}first\PYZus{}value}\PY{p}{(}\PY{n}{portugal}\PY{p}{)} \PY{c+c1}{\PYZsh{}Portuguese population at tp\PYZus{}0}
\end{Verbatim}


    \begin{Verbatim}[commandchars=\\\{\}]
{\color{incolor}In [{\color{incolor}25}]:} \PY{n}{system}\PY{o}{=}\PY{n}{System}\PY{p}{(}\PY{n}{tg\PYZus{}0}\PY{o}{=}\PY{n}{tg\PYZus{}0}\PY{p}{,}       \PY{c+c1}{\PYZsh{}Setting a beginning time range value for Germany}
                      \PY{n}{tg\PYZus{}end}\PY{o}{=}\PY{n}{tg\PYZus{}end}\PY{p}{,}    \PY{c+c1}{\PYZsh{}Setting an end time range for Germany}
                      \PY{n}{pg\PYZus{}0}\PY{o}{=}\PY{n}{pg\PYZus{}0}\PY{p}{,}        \PY{c+c1}{\PYZsh{}Setting a beginning population range value for Germany}
                      \PY{n}{tp\PYZus{}0}\PY{o}{=}\PY{n}{tp\PYZus{}0}\PY{p}{,}        \PY{c+c1}{\PYZsh{}Setting a beginning time range value for Portugal}
                      \PY{n}{tp\PYZus{}end}\PY{o}{=}\PY{n}{tp\PYZus{}end}\PY{p}{,}    \PY{c+c1}{\PYZsh{}Setting an end time range value for Portugal}
                      \PY{n}{pp\PYZus{}0}\PY{o}{=}\PY{n}{pp\PYZus{}0}\PY{p}{,}        \PY{c+c1}{\PYZsh{}Setting a beginning population range value for Portugal}
                      \PY{n}{maxG}\PY{o}{=}\PY{l+m+mi}{85}\PY{p}{,}          \PY{c+c1}{\PYZsh{}Approximate max German population}
                      \PY{n}{maxP}\PY{o}{=}\PY{l+m+mi}{12}\PY{p}{,}          \PY{c+c1}{\PYZsh{}Approximate max Portuguese population}
                      \PY{n}{aG}\PY{o}{=}\PY{o}{.}\PY{l+m+mi}{00035}\PY{p}{,}        \PY{c+c1}{\PYZsh{}Germany\PYZsq{}s population growth rate}
                      \PY{n}{aP}\PY{o}{=}\PY{o}{.}\PY{l+m+mi}{0015}\PY{p}{)}         \PY{c+c1}{\PYZsh{}Portugal\PYZsq{}s population growth rate}
\end{Verbatim}


    Here we create an update function that models each year's population
growth as a function of current population, an assumed maximum
population, and some groth constant aG or aP.

    \begin{Verbatim}[commandchars=\\\{\}]
{\color{incolor}In [{\color{incolor}26}]:} \PY{k}{def} \PY{n+nf}{update\PYZus{}func\PYZus{}quad\PYZus{}germany}\PY{p}{(}\PY{n}{pop}\PY{p}{,} \PY{n}{t}\PY{p}{,} \PY{n}{system}\PY{p}{)}\PY{p}{:}           \PY{c+c1}{\PYZsh{}Defining a general quadratic growth function for Germany}
             \PY{n}{net\PYZus{}growth\PYZus{}germany}\PY{o}{=}\PY{n}{system}\PY{o}{.}\PY{n}{aG}\PY{o}{*}\PY{n}{pop}\PY{o}{*}\PY{p}{(}\PY{n}{system}\PY{o}{.}\PY{n}{maxG}\PY{o}{\PYZhy{}}\PY{n}{pop}\PY{p}{)}  
             \PY{k}{return} \PY{n}{pop}\PY{o}{+}\PY{n}{net\PYZus{}growth\PYZus{}germany}                  
         
         \PY{k}{def} \PY{n+nf}{update\PYZus{}func\PYZus{}quad\PYZus{}portugal}\PY{p}{(}\PY{n}{pop}\PY{p}{,} \PY{n}{t}\PY{p}{,} \PY{n}{system}\PY{p}{)}\PY{p}{:}          \PY{c+c1}{\PYZsh{}Defining a general quadratic growth function for Portugal}
             \PY{n}{net\PYZus{}growth\PYZus{}portugal}\PY{o}{=}\PY{n}{system}\PY{o}{.}\PY{n}{aP}\PY{o}{*}\PY{n}{pop}\PY{o}{*}\PY{p}{(}\PY{n}{system}\PY{o}{.}\PY{n}{maxP}\PY{o}{\PYZhy{}}\PY{n}{pop}\PY{p}{)}
             \PY{k}{return} \PY{n}{pop}\PY{o}{+}\PY{n}{net\PYZus{}growth\PYZus{}portugal}
\end{Verbatim}


    Here we create our run\_simulation functions, which iterate the our
update function through the timeframe for which we have data, and return
the results.

    \begin{Verbatim}[commandchars=\\\{\}]
{\color{incolor}In [{\color{incolor}27}]:} \PY{k}{def} \PY{n+nf}{run\PYZus{}simulation\PYZus{}g}\PY{p}{(}\PY{n}{system}\PY{p}{,} \PY{n}{update\PYZus{}func\PYZus{}g}\PY{p}{)}\PY{p}{:}                  \PY{c+c1}{\PYZsh{}Defining a function that runs simulation for germany}
             
             \PY{n}{resultsg}\PY{o}{=}\PY{n}{TimeSeries}\PY{p}{(}\PY{p}{)}                                     \PY{c+c1}{\PYZsh{}Creating a time series}
             \PY{n}{resultsg}\PY{p}{[}\PY{n}{system}\PY{o}{.}\PY{n}{tg\PYZus{}0}\PY{p}{]}\PY{o}{=}\PY{n}{system}\PY{o}{.}\PY{n}{pg\PYZus{}0}                         \PY{c+c1}{\PYZsh{}Setting the first value in the time series }
             
             \PY{k}{for} \PY{n}{t} \PY{o+ow}{in} \PY{n}{linrange}\PY{p}{(}\PY{n}{system}\PY{o}{.}\PY{n}{tg\PYZus{}0}\PY{p}{,} \PY{n}{system}\PY{o}{.}\PY{n}{tg\PYZus{}end}\PY{p}{)}\PY{p}{:}            \PY{c+c1}{\PYZsh{}Setting the rest of the values in the time series }
                 \PY{n}{resultsg}\PY{p}{[}\PY{n}{t}\PY{o}{+}\PY{l+m+mi}{1}\PY{p}{]} \PY{o}{=} \PY{n}{update\PYZus{}func\PYZus{}g}\PY{p}{(}\PY{n}{resultsg}\PY{p}{[}\PY{n}{t}\PY{p}{]}\PY{p}{,} \PY{n}{t}\PY{p}{,}\PY{n}{system}\PY{p}{)}
                 
             \PY{k}{return} \PY{n}{resultsg}
         
         \PY{k}{def} \PY{n+nf}{run\PYZus{}simulation\PYZus{}p}\PY{p}{(}\PY{n}{system}\PY{p}{,} \PY{n}{update\PYZus{}func\PYZus{}p}\PY{p}{)}\PY{p}{:}                  \PY{c+c1}{\PYZsh{}Defining a function that runs simulation for Portugal}
             
             \PY{n}{resultsp}\PY{o}{=}\PY{n}{TimeSeries}\PY{p}{(}\PY{p}{)}                                     \PY{c+c1}{\PYZsh{}Creating a time series}
             \PY{n}{resultsp}\PY{p}{[}\PY{n}{system}\PY{o}{.}\PY{n}{tp\PYZus{}0}\PY{p}{]}\PY{o}{=}\PY{n}{system}\PY{o}{.}\PY{n}{pp\PYZus{}0}                         \PY{c+c1}{\PYZsh{}Setting the first value in the time series }
             
             \PY{k}{for} \PY{n}{t} \PY{o+ow}{in} \PY{n}{linrange}\PY{p}{(}\PY{n}{system}\PY{o}{.}\PY{n}{tp\PYZus{}0}\PY{p}{,} \PY{n}{system}\PY{o}{.}\PY{n}{tp\PYZus{}end}\PY{p}{)}\PY{p}{:}            \PY{c+c1}{\PYZsh{}Setting the rest of the values in the time series }
                 \PY{n}{resultsp}\PY{p}{[}\PY{n}{t}\PY{o}{+}\PY{l+m+mi}{1}\PY{p}{]} \PY{o}{=} \PY{n}{update\PYZus{}func\PYZus{}p}\PY{p}{(}\PY{n}{resultsp}\PY{p}{[}\PY{n}{t}\PY{p}{]}\PY{p}{,} \PY{n}{t}\PY{p}{,}\PY{n}{system}\PY{p}{)}
                 
             \PY{k}{return} \PY{n}{resultsp}
\end{Verbatim}


    \begin{Verbatim}[commandchars=\\\{\}]
{\color{incolor}In [{\color{incolor}28}]:} \PY{k}{def} \PY{n+nf}{plot\PYZus{}results\PYZus{}g}\PY{p}{(}\PY{n}{real\PYZus{}g}\PY{p}{,} \PY{n}{mod\PYZus{}g}\PY{p}{)}\PY{p}{:}                      \PY{c+c1}{\PYZsh{}Defining a function that plots Germany\PYZsq{}s actual and model populations}
             \PY{n}{plot}\PY{p}{(}\PY{n}{real\PYZus{}g}\PY{p}{,} \PY{l+s+s1}{\PYZsq{}}\PY{l+s+s1}{:}\PY{l+s+s1}{\PYZsq{}}\PY{p}{,} \PY{n}{label}\PY{o}{=}\PY{l+s+s1}{\PYZsq{}}\PY{l+s+s1}{Germany}\PY{l+s+s1}{\PYZsq{}}\PY{p}{)}
             \PY{n}{plot}\PY{p}{(}\PY{n}{mod\PYZus{}g}\PY{p}{,} \PY{n}{label}\PY{o}{=}\PY{l+s+s1}{\PYZsq{}}\PY{l+s+s1}{Germany Model}\PY{l+s+s1}{\PYZsq{}}\PY{p}{)}
             
             \PY{n}{decorate}\PY{p}{(}\PY{n}{xlabel}\PY{o}{=}\PY{l+s+s1}{\PYZsq{}}\PY{l+s+s1}{Year}\PY{l+s+s1}{\PYZsq{}}\PY{p}{,} 
                      \PY{n}{ylabel}\PY{o}{=}\PY{l+s+s1}{\PYZsq{}}\PY{l+s+s1}{Population (millions)}\PY{l+s+s1}{\PYZsq{}}\PY{p}{)}
             
         \PY{k}{def} \PY{n+nf}{plot\PYZus{}results\PYZus{}p}\PY{p}{(}\PY{n}{real\PYZus{}p}\PY{p}{,} \PY{n}{mod\PYZus{}p}\PY{p}{)}\PY{p}{:}                      \PY{c+c1}{\PYZsh{}Defining a function that plots Portugal\PYZsq{}s actual and model populations}
             \PY{n}{plot}\PY{p}{(}\PY{n}{real\PYZus{}p}\PY{p}{,} \PY{l+s+s1}{\PYZsq{}}\PY{l+s+s1}{:}\PY{l+s+s1}{\PYZsq{}}\PY{p}{,} \PY{n}{label}\PY{o}{=}\PY{l+s+s1}{\PYZsq{}}\PY{l+s+s1}{Portugal}\PY{l+s+s1}{\PYZsq{}}\PY{p}{)}
             \PY{n}{plot}\PY{p}{(}\PY{n}{mod\PYZus{}p}\PY{p}{,} \PY{n}{label}\PY{o}{=}\PY{l+s+s1}{\PYZsq{}}\PY{l+s+s1}{Portugal Model}\PY{l+s+s1}{\PYZsq{}}\PY{p}{)}
             
             \PY{n}{decorate}\PY{p}{(}\PY{n}{xlabel}\PY{o}{=}\PY{l+s+s1}{\PYZsq{}}\PY{l+s+s1}{Year}\PY{l+s+s1}{\PYZsq{}}\PY{p}{,} 
                      \PY{n}{ylabel}\PY{o}{=}\PY{l+s+s1}{\PYZsq{}}\PY{l+s+s1}{Population (millions)}\PY{l+s+s1}{\PYZsq{}}\PY{p}{)}
\end{Verbatim}


    \begin{Verbatim}[commandchars=\\\{\}]
{\color{incolor}In [{\color{incolor}29}]:} \PY{n}{resultsg} \PY{o}{=} \PY{n}{run\PYZus{}simulation\PYZus{}g}\PY{p}{(}\PY{n}{system}\PY{p}{,} \PY{n}{update\PYZus{}func\PYZus{}quad\PYZus{}germany}\PY{p}{)}  \PY{c+c1}{\PYZsh{}Runs the function that will plot Germany\PYZsq{}s model}
         
         \PY{n}{plot\PYZus{}results\PYZus{}g}\PY{p}{(}\PY{n}{germany}\PY{p}{,} \PY{n}{resultsg}\PY{p}{)}                              \PY{c+c1}{\PYZsh{}Plots Portugal\PYZsq{}s actual and general quad popuations}
\end{Verbatim}


    \begin{center}
    \adjustimage{max size={0.9\linewidth}{0.9\paperheight}}{output_19_0.png}
    \end{center}
    { \hspace*{\fill} \\}
    
    \begin{Verbatim}[commandchars=\\\{\}]
{\color{incolor}In [{\color{incolor}30}]:} \PY{n}{resultsp} \PY{o}{=} \PY{n}{run\PYZus{}simulation\PYZus{}p}\PY{p}{(}\PY{n}{system}\PY{p}{,} \PY{n}{update\PYZus{}func\PYZus{}quad\PYZus{}portugal}\PY{p}{)}   \PY{c+c1}{\PYZsh{}Runs the function that will plot Portugal\PYZsq{}s model}
         
         \PY{n}{plot\PYZus{}results\PYZus{}p}\PY{p}{(}\PY{n}{portugal}\PY{p}{,} \PY{n}{resultsp}\PY{p}{)}                               \PY{c+c1}{\PYZsh{}Plots Portugal\PYZsq{}s actual and general quad popuations}
\end{Verbatim}


    \begin{center}
    \adjustimage{max size={0.9\linewidth}{0.9\paperheight}}{output_20_0.png}
    \end{center}
    { \hspace*{\fill} \\}
    
    Note that these models were not created by linearizing our data and
fitting a least squares regression line. We eyeballed these models,
because their exact precision was not important.

    \hypertarget{step-3-plot-residuals-to-identify-large-deviations}{%
\subsection{Step 3: Plot residuals to identify large
deviations}\label{step-3-plot-residuals-to-identify-large-deviations}}

At this point, it was already fairly evident where the largest
deviations were from our model, but a relative residual plot would serve
two purposes: - It made it even easier to see deviations from expected
growth - It gave us a means to compare our adjusted model to our generic
model

To calculate a relative residual, we divided the residual from our model
by the true population at the time.

    \begin{Verbatim}[commandchars=\\\{\}]
{\color{incolor}In [{\color{incolor}31}]:} \PY{k}{def} \PY{n+nf}{get\PYZus{}relative\PYZus{}resid}\PY{p}{(}\PY{n}{real}\PY{p}{,} \PY{n}{model}\PY{p}{)}\PY{p}{:}                                  \PY{c+c1}{\PYZsh{}Defines a function that will give us the residual plot }
             \PY{n}{results} \PY{o}{=} \PY{n}{TimeSeries}\PY{p}{(}\PY{p}{)}                                            \PY{c+c1}{\PYZsh{}for the general quadratic model}
             
             \PY{k}{for} \PY{n}{t} \PY{o+ow}{in} \PY{n}{linrange}\PY{p}{(}\PY{n}{get\PYZus{}first\PYZus{}label}\PY{p}{(}\PY{n}{real}\PY{p}{)}\PY{p}{,} \PY{n}{get\PYZus{}last\PYZus{}label}\PY{p}{(}\PY{n}{real}\PY{p}{)}\PY{p}{)}\PY{p}{:}   \PY{c+c1}{\PYZsh{}Takes the difference of the real populations and general}
                 \PY{n}{results}\PY{p}{[}\PY{n}{t}\PY{p}{]} \PY{o}{=} \PY{p}{(}\PY{n}{real}\PY{p}{[}\PY{n}{t}\PY{p}{]} \PY{o}{\PYZhy{}} \PY{n}{model}\PY{p}{[}\PY{n}{t}\PY{p}{]}\PY{p}{)} \PY{o}{/} \PY{n}{real}\PY{p}{[}\PY{n}{t}\PY{p}{]}                   \PY{c+c1}{\PYZsh{}quadratic model and divides by the real population}
                 
             \PY{k}{return} \PY{n}{results}
\end{Verbatim}


    \begin{Verbatim}[commandchars=\\\{\}]
{\color{incolor}In [{\color{incolor}32}]:} \PY{n}{german\PYZus{}model} \PY{o}{=} \PY{n}{run\PYZus{}simulation\PYZus{}g}\PY{p}{(}\PY{n}{system}\PY{p}{,} \PY{n}{update\PYZus{}func\PYZus{}quad\PYZus{}germany}\PY{p}{)}     
         \PY{n}{german\PYZus{}resid} \PY{o}{=} \PY{n}{get\PYZus{}relative\PYZus{}resid}\PY{p}{(}\PY{n}{germany}\PY{p}{,} \PY{n}{german\PYZus{}model}\PY{p}{)}           \PY{c+c1}{\PYZsh{}Creates Portugal\PYZsq{}s residual model for general quadratic model}
         
         \PY{n}{plot}\PY{p}{(}\PY{n}{german\PYZus{}resid}\PY{p}{,} \PY{n}{label} \PY{o}{=} \PY{l+s+s1}{\PYZsq{}}\PY{l+s+s1}{Relative Residuals for Germany}\PY{l+s+s1}{\PYZsq{}}\PY{p}{)}       \PY{c+c1}{\PYZsh{}Plots Germany\PYZsq{}s residual plot for the general quadratic model}
         
         \PY{n}{decorate}\PY{p}{(}\PY{n}{xlabel} \PY{o}{=} \PY{l+s+s1}{\PYZsq{}}\PY{l+s+s1}{Time (years)}\PY{l+s+s1}{\PYZsq{}}\PY{p}{,} \PY{n}{ylabel} \PY{o}{=} \PY{l+s+s1}{\PYZsq{}}\PY{l+s+s1}{Relative Residual}\PY{l+s+s1}{\PYZsq{}}\PY{p}{)}
\end{Verbatim}


    \begin{center}
    \adjustimage{max size={0.9\linewidth}{0.9\paperheight}}{output_24_0.png}
    \end{center}
    { \hspace*{\fill} \\}
    
    \begin{Verbatim}[commandchars=\\\{\}]
{\color{incolor}In [{\color{incolor}33}]:} \PY{n}{portuguese\PYZus{}model} \PY{o}{=} \PY{n}{run\PYZus{}simulation\PYZus{}p}\PY{p}{(}\PY{n}{system}\PY{p}{,} \PY{n}{update\PYZus{}func\PYZus{}quad\PYZus{}portugal}\PY{p}{)}
         \PY{n}{portuguese\PYZus{}resid} \PY{o}{=} \PY{n}{get\PYZus{}relative\PYZus{}resid}\PY{p}{(}\PY{n}{portugal}\PY{p}{,} \PY{n}{portuguese\PYZus{}model}\PY{p}{)} \PY{c+c1}{\PYZsh{}Creates Portugal\PYZsq{}s residual model for general quadratic model}
         
         \PY{n}{plot}\PY{p}{(}\PY{n}{portuguese\PYZus{}resid}\PY{p}{,} \PY{n}{label} \PY{o}{=} \PY{l+s+s1}{\PYZsq{}}\PY{l+s+s1}{Relative Residuals for Portugal}\PY{l+s+s1}{\PYZsq{}}\PY{p}{)} \PY{c+c1}{\PYZsh{}Plots Portugal\PYZsq{}s residual plot for the general quadratic model}
         
         \PY{n}{decorate}\PY{p}{(}\PY{n}{xlabel} \PY{o}{=} \PY{l+s+s1}{\PYZsq{}}\PY{l+s+s1}{Time (years)}\PY{l+s+s1}{\PYZsq{}}\PY{p}{,} \PY{n}{ylabel} \PY{o}{=} \PY{l+s+s1}{\PYZsq{}}\PY{l+s+s1}{Relative Residual}\PY{l+s+s1}{\PYZsq{}}\PY{p}{)}
\end{Verbatim}


    \begin{center}
    \adjustimage{max size={0.9\linewidth}{0.9\paperheight}}{output_25_0.png}
    \end{center}
    { \hspace*{\fill} \\}
    
    \hypertarget{step-4-research}{%
\subsection{Step 4: Research!}\label{step-4-research}}

Now that we have identified 1970, and the buildup to that date, as a
significant deviation from expected population, we knew exactly where to
look for population events. A quick google search for `german population
boom 1970' led us to this study: http://countrystudies.us/germany/84.htm
which we then verified with this New York Times article:
https://www.nytimes.com/1984/08/19/magazine/germany-s-guest-workers.html

    \hypertarget{gastarbeiter}{%
\subsubsection{Gastarbeiter}\label{gastarbeiter}}

We found that due to Germany's booming economy in the 1950's, many
workers were brought in to work in a variety of industries, including
infrastructure. Between 1955 and 1973, seven countries, including
Portugal, entered agreements with Germany which made it easier for
workers to travel to Germany for work. These workers, called
gastarbeiter, were originally meant to stay only for 3 years, though
many of them stayed in Germany permanently.

    \hypertarget{step-5-a-new-model}{%
\subsection{Step 5: A New Model}\label{step-5-a-new-model}}

    Knowing the Gastarbeiter migration event, we created a model for each of
the countries with the expectation of population growth and decline. The
first step for creating this new model was coding a function of the rate
of migrants flowing into Germany.

    \begin{Verbatim}[commandchars=\\\{\}]
{\color{incolor}In [{\color{incolor}34}]:} \PY{k}{def} \PY{n+nf}{travel\PYZus{}popularity}\PY{p}{(}\PY{n}{t}\PY{p}{)}\PY{p}{:}       \PY{c+c1}{\PYZsh{}Defines the function of migrants flowing into Germany}
             \PY{n}{pop} \PY{o}{=}  \PY{o}{\PYZhy{}}\PY{l+m+mi}{1}\PY{o}{*}\PY{p}{(}\PY{n}{t}\PY{o}{\PYZhy{}}\PY{l+m+mi}{1960}\PY{p}{)}\PY{o}{*}\PY{p}{(}\PY{n}{t}\PY{o}{\PYZhy{}}\PY{l+m+mi}{1973}\PY{p}{)} \PY{c+c1}{\PYZsh{}5 year delay for program to become significantly popular}
             \PY{k}{if} \PY{n}{t} \PY{o}{\PYZlt{}}\PY{o}{=} \PY{l+m+mi}{1960} \PY{o+ow}{or} \PY{n}{t} \PY{o}{\PYZgt{}}\PY{o}{=} \PY{l+m+mi}{1973}\PY{p}{:}  \PY{c+c1}{\PYZsh{}Sets the time frame}
                 \PY{k}{return} \PY{l+m+mi}{0}
             \PY{k}{else}\PY{p}{:}
                 \PY{k}{return} \PY{n}{pop}
\end{Verbatim}


    \begin{Verbatim}[commandchars=\\\{\}]
{\color{incolor}In [{\color{incolor}35}]:} \PY{n}{results} \PY{o}{=} \PY{n}{TimeSeries}\PY{p}{(}\PY{p}{)}                \PY{c+c1}{\PYZsh{}Creates the time series }
         \PY{n}{time} \PY{o}{=} \PY{n}{linrange}\PY{p}{(}\PY{l+m+mi}{1955}\PY{p}{,} \PY{l+m+mi}{1974}\PY{p}{)}          
         \PY{k}{for} \PY{n}{t} \PY{o+ow}{in} \PY{n}{time}\PY{p}{:}                        \PY{c+c1}{\PYZsh{}Populates TimeSeries with popularity of migrant program}
             \PY{n}{results}\PY{p}{[}\PY{n}{t}\PY{p}{]} \PY{o}{=} \PY{n}{travel\PYZus{}popularity}\PY{p}{(}\PY{n}{t}\PY{p}{)} 
             
         \PY{n}{plot}\PY{p}{(}\PY{n}{results}\PY{p}{)}
\end{Verbatim}


\begin{Verbatim}[commandchars=\\\{\}]
{\color{outcolor}Out[{\color{outcolor}35}]:} [<matplotlib.lines.Line2D at 0x1fad16270b8>]
\end{Verbatim}
            
    \begin{center}
    \adjustimage{max size={0.9\linewidth}{0.9\paperheight}}{output_31_1.png}
    \end{center}
    { \hspace*{\fill} \\}
    
    This function starts off at 1960 rather than the actual start date of
1955 because we predicted that the program did not get popular enough to
significantly affect the population. The shape of the graph is then a
parabolic curve because the popularity would highten in the middle years
of the program and then slowly decrease to 0 as the German economy
suffered in the early 70's.

    We made a new system to account for new variables.

    \begin{Verbatim}[commandchars=\\\{\}]
{\color{incolor}In [{\color{incolor}36}]:} \PY{n}{system}\PY{o}{=}\PY{n}{System}\PY{p}{(}\PY{n}{tg\PYZus{}0}\PY{o}{=}\PY{n}{tg\PYZus{}0}\PY{p}{,}        \PY{c+c1}{\PYZsh{}Setting a beginning time range value for Germany}
                      \PY{n}{tg\PYZus{}end}\PY{o}{=}\PY{n}{tg\PYZus{}end}\PY{p}{,}     \PY{c+c1}{\PYZsh{}Setting an end time range for Germany}
                      \PY{n}{pg\PYZus{}0}\PY{o}{=}\PY{n}{pg\PYZus{}0}\PY{p}{,}         \PY{c+c1}{\PYZsh{}Setting a beginning population range value for Germany}
                      \PY{n}{tp\PYZus{}0}\PY{o}{=}\PY{n}{tp\PYZus{}0}\PY{p}{,}         \PY{c+c1}{\PYZsh{}Setting a beginning time range value for Portugal}
                      \PY{n}{tp\PYZus{}end}\PY{o}{=}\PY{n}{tp\PYZus{}end}\PY{p}{,}     \PY{c+c1}{\PYZsh{}Setting an end time range value for Portugal}
                      \PY{n}{pp\PYZus{}0}\PY{o}{=}\PY{n}{pp\PYZus{}0}\PY{p}{,}         \PY{c+c1}{\PYZsh{}Setting a beginning population range value for Portugal}
                      \PY{n}{maxG}\PY{o}{=}\PY{l+m+mi}{85}\PY{p}{,}           \PY{c+c1}{\PYZsh{}Approximate maximum population of Germany}
                      \PY{n}{maxP}\PY{o}{=}\PY{l+m+mi}{12}\PY{p}{,}           \PY{c+c1}{\PYZsh{}Approximate maximum population of Germany}
                      \PY{n}{aG}\PY{o}{=}\PY{o}{.}\PY{l+m+mi}{00035}\PY{p}{,}         \PY{c+c1}{\PYZsh{}Germany\PYZsq{}s population growth rate}
                      \PY{n}{aP}\PY{o}{=}\PY{o}{.}\PY{l+m+mi}{0015}\PY{p}{,}          \PY{c+c1}{\PYZsh{}Portugal\PYZsq{}s population growth rate}
                      \PY{n}{c}\PY{o}{=}\PY{l+m+mf}{0.015}\PY{p}{,}           \PY{c+c1}{\PYZsh{}Popularity of migrant program}
                      \PY{n}{stayrate}\PY{o}{=}\PY{l+m+mi}{2}\PY{o}{/}\PY{l+m+mi}{3}\PY{p}{,}      \PY{c+c1}{\PYZsh{}the fraction of migrants who stayed in Germany after 3 years}
                      \PY{n}{propport}\PY{o}{=}\PY{l+m+mi}{1}\PY{o}{/}\PY{l+m+mi}{5}\PY{p}{)}      \PY{c+c1}{\PYZsh{}the fraction of migrants coming to Germany from Portugal}
\end{Verbatim}


    To create our new model for Germany, we added our travel\_popularity
function to our generic quadratic model. We also accounted for 1/3 of
migrants leaving Germany after 3 years.

    \begin{Verbatim}[commandchars=\\\{\}]
{\color{incolor}In [{\color{incolor}37}]:} \PY{k}{def} \PY{n+nf}{update\PYZus{}func\PYZus{}migrant\PYZus{}germany}\PY{p}{(}\PY{n}{pop}\PY{p}{,} \PY{n}{t}\PY{p}{,} \PY{n}{system}\PY{p}{)}\PY{p}{:}          \PY{c+c1}{\PYZsh{}Creates a function that adds travel\PYZus{}popularity * a constant to }
                                                                   \PY{c+c1}{\PYZsh{}Germany\PYZsq{}s general quadratic function}
             \PY{n}{net\PYZus{}growth\PYZus{}germany} \PY{o}{=} \PY{n}{system}\PY{o}{.}\PY{n}{aG}\PY{o}{*}\PY{n}{pop}\PY{o}{*}\PY{p}{(}\PY{n}{system}\PY{o}{.}\PY{n}{maxG}\PY{o}{\PYZhy{}}\PY{n}{pop}\PY{p}{)}
             \PY{n}{net\PYZus{}growth\PYZus{}germany} \PY{o}{+}\PY{o}{=} \PY{n}{travel\PYZus{}popularity}\PY{p}{(}\PY{n}{t}\PY{p}{)}\PY{o}{*}\PY{n}{system}\PY{o}{.}\PY{n}{c}
             \PY{n}{net\PYZus{}growth\PYZus{}germany} \PY{o}{\PYZhy{}}\PY{o}{=} \PY{n}{travel\PYZus{}popularity}\PY{p}{(}\PY{n}{t}\PY{o}{\PYZhy{}}\PY{l+m+mi}{3}\PY{p}{)}\PY{o}{*}\PY{n}{system}\PY{o}{.}\PY{n}{c}\PY{o}{*}\PY{n}{system}\PY{o}{.}\PY{n}{stayrate} \PY{c+c1}{\PYZsh{}and subtracts 1/3 of travel\PYZus{}popularity after 3 years}
             \PY{c+c1}{\PYZsh{}Research shows \PYZti{}2/3 of people stayed in the country bc of the booming economy}
             
             \PY{k}{return} \PY{n}{pop}\PY{o}{+}\PY{n}{net\PYZus{}growth\PYZus{}germany} 
\end{Verbatim}


    \begin{Verbatim}[commandchars=\\\{\}]
{\color{incolor}In [{\color{incolor}38}]:} \PY{n}{resultsg} \PY{o}{=} \PY{n}{run\PYZus{}simulation\PYZus{}g}\PY{p}{(}\PY{n}{system}\PY{p}{,} \PY{n}{update\PYZus{}func\PYZus{}migrant\PYZus{}germany}\PY{p}{)}
         
         \PY{n}{plot\PYZus{}results\PYZus{}g}\PY{p}{(}\PY{n}{germany}\PY{p}{,} \PY{n}{resultsg}\PY{p}{)}                       \PY{c+c1}{\PYZsh{}Plots Germany\PYZsq{}s new model alongside actual population}
\end{Verbatim}


    \begin{center}
    \adjustimage{max size={0.9\linewidth}{0.9\paperheight}}{output_37_0.png}
    \end{center}
    { \hspace*{\fill} \\}
    
    To create our new model for Portugal we subtracted people from the
generic model according to the travel\_popularity fuction. We also added
back the 1/3 of the workers who returned to Portugal. We also accounted
for the fact that Portugal was one of seven countries from which
migrants went to Germany by multiplying the function by 1/5. This
assumes that 1/5 of the migrants in Germany came from Portugal.

    \begin{Verbatim}[commandchars=\\\{\}]
{\color{incolor}In [{\color{incolor}39}]:} \PY{k}{def} \PY{n+nf}{update\PYZus{}func\PYZus{}migrant\PYZus{}portugal}\PY{p}{(}\PY{n}{pop}\PY{p}{,} \PY{n}{t}\PY{p}{,} \PY{n}{system}\PY{p}{)}\PY{p}{:}         \PY{c+c1}{\PYZsh{}Creates a function that subtracts travel\PYZus{}popularity by a constant to }
                                                                   \PY{c+c1}{\PYZsh{}Portugal\PYZsq{}s general quadratic function}
             
             \PY{n}{net\PYZus{}growth\PYZus{}portugal} \PY{o}{=} \PY{n}{system}\PY{o}{.}\PY{n}{aP}\PY{o}{*}\PY{n}{pop}\PY{o}{*}\PY{p}{(}\PY{n}{system}\PY{o}{.}\PY{n}{maxP}\PY{o}{\PYZhy{}}\PY{n}{pop}\PY{p}{)}
             \PY{n}{net\PYZus{}growth\PYZus{}portugal} \PY{o}{\PYZhy{}}\PY{o}{=} \PY{n}{travel\PYZus{}popularity}\PY{p}{(}\PY{n}{t}\PY{p}{)}\PY{o}{*}\PY{n}{system}\PY{o}{.}\PY{n}{c}\PY{o}{*}\PY{n}{system}\PY{o}{.}\PY{n}{propport}
             \PY{n}{net\PYZus{}growth\PYZus{}portugal} \PY{o}{+}\PY{o}{=} \PY{n}{travel\PYZus{}popularity}\PY{p}{(}\PY{n}{t}\PY{o}{\PYZhy{}}\PY{l+m+mi}{3}\PY{p}{)}\PY{o}{*}\PY{n}{system}\PY{o}{.}\PY{n}{c}\PY{o}{*}\PY{n}{system}\PY{o}{.}\PY{n}{stayrate}\PY{o}{*}\PY{n}{system}\PY{o}{.}\PY{n}{propport} \PY{c+c1}{\PYZsh{}and adds 1/3 of travel\PYZus{}popularity}
                                                                                                    \PY{c+c1}{\PYZsh{} after 3 years}
             \PY{c+c1}{\PYZsh{}Portugal was one of seven countries sending workers to Germany}
             
             \PY{k}{return} \PY{n}{pop}\PY{o}{+}\PY{n}{net\PYZus{}growth\PYZus{}portugal}
\end{Verbatim}


    \begin{Verbatim}[commandchars=\\\{\}]
{\color{incolor}In [{\color{incolor}40}]:} \PY{n}{resultsp} \PY{o}{=} \PY{n}{run\PYZus{}simulation\PYZus{}p}\PY{p}{(}\PY{n}{system}\PY{p}{,} \PY{n}{update\PYZus{}func\PYZus{}migrant\PYZus{}portugal}\PY{p}{)}
         
         \PY{n}{plot\PYZus{}results\PYZus{}p}\PY{p}{(}\PY{n}{portugal}\PY{p}{,} \PY{n}{resultsp}\PY{p}{)}                                  \PY{c+c1}{\PYZsh{}Plots Portugal\PYZsq{}s new model alongside actual population}
\end{Verbatim}


    \begin{center}
    \adjustimage{max size={0.9\linewidth}{0.9\paperheight}}{output_40_0.png}
    \end{center}
    { \hspace*{\fill} \\}
    
    To see how our new model compared with our generic model, we made two
residual plots that noted the difference between the our models and the
actual data.

    \begin{Verbatim}[commandchars=\\\{\}]
{\color{incolor}In [{\color{incolor}41}]:} \PY{n}{german\PYZus{}model\PYZus{}quad} \PY{o}{=} \PY{n}{run\PYZus{}simulation\PYZus{}g}\PY{p}{(}\PY{n}{system}\PY{p}{,} \PY{n}{update\PYZus{}func\PYZus{}quad\PYZus{}germany}\PY{p}{)}
         \PY{n}{german\PYZus{}model\PYZus{}migrant} \PY{o}{=} \PY{n}{run\PYZus{}simulation\PYZus{}g}\PY{p}{(}\PY{n}{system}\PY{p}{,} \PY{n}{update\PYZus{}func\PYZus{}migrant\PYZus{}germany}\PY{p}{)}   \PY{c+c1}{\PYZsh{}Creates residual model for Germany\PYZsq{}s quad model}
         
         \PY{n}{german\PYZus{}resid\PYZus{}quad} \PY{o}{=} \PY{n}{get\PYZus{}relative\PYZus{}resid}\PY{p}{(}\PY{n}{germany}\PY{p}{,} \PY{n}{german\PYZus{}model\PYZus{}quad}\PY{p}{)}
         \PY{n}{german\PYZus{}resid\PYZus{}migrant} \PY{o}{=} \PY{n}{get\PYZus{}relative\PYZus{}resid}\PY{p}{(}\PY{n}{germany}\PY{p}{,} \PY{n}{german\PYZus{}model\PYZus{}migrant}\PY{p}{)}       \PY{c+c1}{\PYZsh{}Creates residual model for Germany\PYZsq{}s new model}
         
         
         \PY{n}{plot}\PY{p}{(}\PY{n}{german\PYZus{}resid\PYZus{}quad}\PY{p}{,} \PY{n}{label}\PY{o}{=}\PY{l+s+s1}{\PYZsq{}}\PY{l+s+s1}{Quadratic Model}\PY{l+s+s1}{\PYZsq{}}\PY{p}{)}                               \PY{c+c1}{\PYZsh{}Plots residuals of Germany\PYZsq{}s quadratic model}
         \PY{n}{plot}\PY{p}{(}\PY{n}{german\PYZus{}resid\PYZus{}migrant}\PY{p}{,} \PY{n}{label}\PY{o}{=}\PY{l+s+s1}{\PYZsq{}}\PY{l+s+s1}{Migrant Model}\PY{l+s+s1}{\PYZsq{}}\PY{p}{)}                              \PY{c+c1}{\PYZsh{}Plots residuals of Germany\PYZsq{}s new model}
         
         \PY{n}{decorate}\PY{p}{(}\PY{n}{xlabel}\PY{o}{=}\PY{l+s+s1}{\PYZsq{}}\PY{l+s+s1}{Year}\PY{l+s+s1}{\PYZsq{}}\PY{p}{,} \PY{n}{ylabel}\PY{o}{=}\PY{l+s+s1}{\PYZsq{}}\PY{l+s+s1}{Relative Residual}\PY{l+s+s1}{\PYZsq{}}\PY{p}{,} \PY{n}{title}\PY{o}{=}\PY{l+s+s1}{\PYZsq{}}\PY{l+s+s1}{Residuals}\PY{l+s+s1}{\PYZsq{}}\PY{p}{)}\PY{p}{;}
\end{Verbatim}


    \begin{center}
    \adjustimage{max size={0.9\linewidth}{0.9\paperheight}}{output_42_0.png}
    \end{center}
    { \hspace*{\fill} \\}
    
    \begin{Verbatim}[commandchars=\\\{\}]
{\color{incolor}In [{\color{incolor}42}]:} \PY{n}{portuguese\PYZus{}model} \PY{o}{=} \PY{n}{run\PYZus{}simulation\PYZus{}p}\PY{p}{(}\PY{n}{system}\PY{p}{,} \PY{n}{update\PYZus{}func\PYZus{}quad\PYZus{}portugal}\PY{p}{)}          
         \PY{n}{portuguese\PYZus{}resid\PYZus{}quad} \PY{o}{=} \PY{n}{get\PYZus{}relative\PYZus{}resid}\PY{p}{(}\PY{n}{portugal}\PY{p}{,} \PY{n}{portuguese\PYZus{}model}\PY{p}{)}          \PY{c+c1}{\PYZsh{}Creates residual model for Portugal\PYZsq{}s quad model}
         
         \PY{n}{portuguese\PYZus{}model} \PY{o}{=} \PY{n}{run\PYZus{}simulation\PYZus{}p}\PY{p}{(}\PY{n}{system}\PY{p}{,} \PY{n}{update\PYZus{}func\PYZus{}migrant\PYZus{}portugal}\PY{p}{)}
         \PY{n}{portuguese\PYZus{}resid\PYZus{}migrant} \PY{o}{=} \PY{n}{get\PYZus{}relative\PYZus{}resid}\PY{p}{(}\PY{n}{portugal}\PY{p}{,} \PY{n}{portuguese\PYZus{}model}\PY{p}{)}       \PY{c+c1}{\PYZsh{}Creates residual model for Portugal\PYZsq{}s new model}
         
         \PY{n}{plot}\PY{p}{(}\PY{n}{portuguese\PYZus{}resid\PYZus{}quad}\PY{p}{,} \PY{n}{label}\PY{o}{=}\PY{l+s+s1}{\PYZsq{}}\PY{l+s+s1}{Quadratic Model}\PY{l+s+s1}{\PYZsq{}}\PY{p}{)}                            \PY{c+c1}{\PYZsh{}Plots residuals of Portugal\PYZsq{}s quadratic model}
         \PY{n}{plot}\PY{p}{(}\PY{n}{portuguese\PYZus{}resid\PYZus{}migrant}\PY{p}{,} \PY{n}{label}\PY{o}{=}\PY{l+s+s1}{\PYZsq{}}\PY{l+s+s1}{Migrant Model}\PY{l+s+s1}{\PYZsq{}}\PY{p}{)}                           \PY{c+c1}{\PYZsh{}Plots residuals of Portugal\PYZsq{}s new model}
         
         \PY{n}{decorate}\PY{p}{(}\PY{n}{xlabel}\PY{o}{=}\PY{l+s+s1}{\PYZsq{}}\PY{l+s+s1}{Year}\PY{l+s+s1}{\PYZsq{}}\PY{p}{,} \PY{n}{ylabel}\PY{o}{=}\PY{l+s+s1}{\PYZsq{}}\PY{l+s+s1}{Relative Residual}\PY{l+s+s1}{\PYZsq{}}\PY{p}{,} \PY{n}{title}\PY{o}{=}\PY{l+s+s1}{\PYZsq{}}\PY{l+s+s1}{Residuals}\PY{l+s+s1}{\PYZsq{}}\PY{p}{)}\PY{p}{;}
\end{Verbatim}


    \begin{center}
    \adjustimage{max size={0.9\linewidth}{0.9\paperheight}}{output_43_0.png}
    \end{center}
    { \hspace*{\fill} \\}
    
    \hypertarget{interpretation}{%
\subsection{Interpretation:}\label{interpretation}}

\begin{itemize}
\tightlist
\item
  The quadratic growth model roughly approximated the population curve,
  though there were clearly variations in the population that were not
  accounted for in that model
\item
  This growth model, along with its residual plot, showed us when
  significant deviations in populations occurred
\item
  The new model almost perfectly accounted for the change in Germany's
  population, and accounted for over half of the population decrease in
  Portugal
\item
  Our migrant model was successful because in the time range 1955-1973,
  the residuals were significantly lower than those of a generic
  logistic model
\end{itemize}

A few causes of error include:

\begin{itemize}
\tightlist
\item
  We were unsure how many migrants workers stayed in Germany
\item
  We were unsure how the popularity of migration changed with time
\item
  We were unsure what percentage of the migrant workers in Germany came
  from Portugal
\end{itemize}

    \hypertarget{sources}{%
\subsection{Sources}\label{sources}}

\begin{itemize}
\tightlist
\item
  https://www.gapminder.org/data/
\item
  http://countrystudies.us/germany/84.htm
\item
  https://www.nytimes.com/1984/08/19/magazine/germany-s-guest-workers.html
\end{itemize}


    % Add a bibliography block to the postdoc
    
    
    
    \end{document}
